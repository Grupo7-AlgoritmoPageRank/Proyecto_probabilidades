\documentclass[twocolumn]{article}
%\documentclass{article}
% Margenes----------------------------------------------------------
\textheight = 24cm
\textwidth = 18.5cm % Ancho
\topmargin = -2.5cm
\oddsidemargin = -1cm
\parindent = 0mm % Sin sangr????a
%Paquetes adicionales-----------------------------------------------
%Otra opcin para m????rgenes,etc., es el paquete geometry.
%\usepackage[total={18cm,21cm},top=2cm, left=2cm]{geometry}
\usepackage{latexsym,amsmath,amssymb,amsfonts}
%\usepackage[latin1]{inputenc}
\usepackage[T1]{fontenc}
\usepackage{graphicx}
\usepackage[spanish, activeacute]{babel} %Definir idioma español
\usepackage[utf8]{inputenc} %Codificacion utf-8
%\usepackage[lined,boxed]{algorithm2e}
%\usepackage[linesnumbered]{algorithm2e}

\usepackage[linesnumbered,ruled,vlined]{algorithm2e}
\usepackage{algorithmic}
%\usepackage[spanish]{babel} % Idioma espa????ol
\renewcommand{\baselinestretch}{1.1} % espaciado 1.1
\pagestyle{myheadings}

% \markright{...... texto .......}
%-------------------------------------------------------------------
\begin{document} 

%titulo
\title{
Cálculo del ranking de los países sudamericanos que participaron en las eliminatorias para el mundial Rusia 2018 usando el algoritmo PageRank
}
\date{}
\author{Alex Avila, Bazalar Rommel,Konrad Trejo\\
\small{\texttt{\@aavilas@uni.pe, rbazalarc@uni.pe,konrad.trejo.c@uni.pe}}}
\maketitle

%Seccion Introduccion
\section{Introducción}

	El objetivo de este proyecto es lograr un ranking parecido al de FIFA, pero solo con las selecciones sudamericanas tomando en cuenta los partidos de la eliminatoria para el mundial Rusia 2018.

	Con el fin de cumplir el objetivo mencionado, surge la idea de la creación de un nuevo ranking basándonos en el método de PageRank (generado por Larry Page y Sergey Brin), generando así un nuevo ranking en que se destacan los países sudamericanos.

	Basándonos en un historial de los partidos, nos proponemos evaluar cómo se comporta el nuevo método creado en comparación de otros rankings existentes y así observar cual de los métodos disponibles refleja mejor la realidad.

	Una vez obtenido el ranking, nos proponemos a, dado un partido, evaluar el ranking de los países que lo disputan  así poder indicar con que probabilidad saldrá victorioso el equipo con mejor posicionamiento. Para lograr todo esto tendrémos que usar nuestros conocimientos en el curso de Probabilidades y también aprovechar nuestro conocimiento en el lenguaje R.

% La bibliografia se encuentra de el archivo 
 %bibliografia.bib adjunto en los\archivos

%Seccion estado del arte
\section{Estado del arte}
\begin{itemize}
	\item
	"TenisRank: Un Nuevo Ranking De Jugadores De Tenis
basado en PageRank", Analizan un nuevo ranking de jugadores de tenis debido a las críticas existentes del ranking actual. Para esto utilizan el famoso algoritmo de google "PageRank".
	\item
	"PageRank Approach to Ranking National Football Teams". Analizan los datos disponibles sobre los campeonatos mundiales de fútbol desde 1930 hasta hoy. El objetivo del paper es clasificar los equipos nacionales en función de todos los partidos durante los campeonatos. Para este propósito, aplican el PageRank creado a partir de los partidos jugados durante los torneos. 
	\end{itemize}	
%seccion Diseño del experimento
\section{Diseño del experimento}
\subsection{Objetos, funciones y técnicas utilizadas}
\begin{itemize}
	\item
	Vectores : Son una de las unidades básicas de trabajo en R y muchos de los métodos de trabajo con este tipo de objetos se aplica para otras clases deobjetos. Por ejemplo, casi todos los objetos pueden ser indexados usando [ ].
	Un vector es una colección de uno o más objetos del mismo tipo (
caracteres, números,etc); esta es una restricción importante a la hora de crear un vector, como veremos más adelante.
	\item
	Matrices : Una matrix es un array con dos dimensiones. Tienen una
funcionalidad muy parecida a la del vector.
	\item
    Plot  : La función plot es una función genérica para la representación gráfica de objetos en R. Los gráficos más sencillos que permite generar esta función son nubes de puntos (x,y).
    \item 
    Grafos con Igraph : El paquete para R Igraph, necesita que se le presenten los datos de la matriz de adyacencia por parejas. Es decir, una matriz de doble entrada convencional (también llamada sociomatriz, tabla de confundido o tabla de concordancia) ha de pasarse al formato de Igraph.
    \item
    Eigen : Calcula valores propios y vectores propios de matrices numéricas (dobles, enteras, lógicas) o complejas.\\ \\ \\ \\ \\ \\
	\end{itemize}	
%Subseccion Experimentos y resultados
\section{Experimentos y resultados}

\begin{itemize}
	\item
Línea base :\\ \\
    \begin{tabular}{|l|l|}
	\hline
País & Ranking por FIFA \\
\hline \hline
Brasil & 1\\ \hline
Uruguay & 2 \\ \hline
Argentina & 3 \\ \hline
Colombia & 4 \\ \hline
Perú & 5 \\ \hline
Chile & 6 \\ \hline
Paraguay & 7 \\ \hline
Ecuador & 8 \\ \hline
Venezuela & 9 \\ \hline
Bolivia & 10 \\ \hline
\end{tabular}
	\item
Evaluación del rendimiento de los modelos ensayados: \\

	\begin{tabular}{|l|l|l|}
	\hline
Ranking & País  & PageRank\\
\hline \hline \hline
1&Brasil & 0.4738853  \\ \hline
2&Uruguay & 0.3731390  \\ \hline
3&Argentina & 0.3793522  \\ \hline
4&Colombia & 0.3153363  \\ \hline
5&Perú & 0.2846076  \\ \hline
6&Chile & 0.2163052 \\ \hline
7&Paraguay & 0.2669837  \\ \hline
8&Ecuador & 0.1977172 \\ \hline
9&Venezuela & 0.1546235 \\ \hline
10&Bolivia & 0.3616417  \\ \hline

\end{tabular}
    \item
    Comparación de línea base y resultados propios :\\ 
 Podemos ver que el primer puesto lo ocupa Brasil en ambos ranking's. El segundo puesto y tercer puesto sí han cambiado con respecto del Ranking FIFA, en el que Uruguay y Argentina los ocupan respectivamente; en cambio, con el algoritmo PageRank, Argentina y Uruguay ocupan esos puestos respectivamente. En el cuarto puesto sí hay un gran cambio pues Bolivia ascendió del último puesto es Ranking FIFA al cuarto en el PageRank, otro cambio más sería que Paraguay está por encima de Chile en el nuevo Ranking desarrollado con PageRank .
	\end{itemize}	

%\subsection{Ejemplo de subsecuencia}

%Ejemplo de secuencias $X_{1}$,$X_{2}$, . . ., $X_{m}$

%Ejemplo de sumatoria
%\[O(
%\sum_{k=1}^{n}p_{k}log(1/p_{k}))
%\]
% Ejemplo de O grande
%\begin{center}
% O(log($min_{i<j}$(|$x_{i}$| + $t_{i}$ + 2)))    
%\end{center}

\section{Discusión}
 Interpretación de los resultados obtenidos.
\begin{itemize}
\item Puede ser que el algoritmo basado en PageRank sea congruente con el de la FIFA cuando se trata de no dar falsos positivos de equipos de baja categoría; sin embargo, como es el caso de Bolivia, un falso positivo de un equipo de alta categoría es posible.

\end{itemize}

¿Cómo podría ser mejorado sus resultados?
\begin{itemize}
\item Una mayor cantidad de datos mejoraría la precisión de nuestros resultados, obviamente; aún así, es importante señalar que actualmente no se puede realizar un análisis completo de la eficacia del algoritmo, dado que la muestra de todos los equipos sudamericanos es muy pequeña, por lo tanto se necesitaría su desempeño a través de los años en más-de-una copas mundiales para compensar.

\end{itemize}
 
%\begin{figure}

%	\includegraphics[width=0.5\textwidth]{zigzag.png}
%	\caption{Movimientos zig-zig y zig-zag}\label{FI1_2}
%\end{figure}


%\begin{figure}[h!]
%	\centering
%	\includegraphics[width=0.5\textwidth]{toques.png}
%	\caption{Gr\'afica de elementos del \'arbol versus recorridos por el algoritmo al buscarlos}\label{FI1_3}
%\end{figure}


\section{Conclusiones y trabajos futuros}


\section{Bibliografía}
\begin{itemize}
\item http://es.fifa.com/worldcup/preliminaries/index.html
\item 
M. Eisermann, Comment Google classe les pages webb,
http://images.math.cnrs.fr/Comment-Google-classe-les-pages.html, 2009.
\item 
A.N. Langville and C. D. Meyer, A Survey of Eigenvector Methods
for Web Information Retrieval, SIAM Review, Volume 47, Issue 1,
(2005), pp. 135–161.
\end{itemize}
%Fin del documento
\end{document}