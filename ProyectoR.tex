\documentclass[twocolumn]{article}
%\documentclass{article}
% Margenes----------------------------------------------------------
\textheight = 24cm
\textwidth = 18.5cm % Ancho
\topmargin = -2.5cm
\oddsidemargin = -1cm
\parindent = 0mm % Sin sangr????a
%Paquetes adicionales-----------------------------------------------
%Otra opcin para m????rgenes,etc., es el paquete geometry.
%\usepackage[total={18cm,21cm},top=2cm, left=2cm]{geometry}
\usepackage{latexsym,amsmath,amssymb,amsfonts}
%\usepackage[latin1]{inputenc}
\usepackage[T1]{fontenc}
\usepackage{graphicx}
\usepackage[spanish, activeacute]{babel} %Definir idioma español
\usepackage[utf8]{inputenc} %Codificacion utf-8
%\usepackage[lined,boxed]{algorithm2e}
%\usepackage[linesnumbered]{algorithm2e}

\usepackage[linesnumbered,ruled,vlined]{algorithm2e}
\usepackage{algorithmic}
%\usepackage[spanish]{babel} % Idioma espa????ol
\renewcommand{\baselinestretch}{1.1} % espaciado 1.1
\pagestyle{myheadings}

% \markright{...... texto .......}
%-------------------------------------------------------------------
\begin{document} 

%titulo
\title{
Clasificaci\'on de los equipos nacionales de f\'utbol   usando el algoritmo PageRank
}
\date{}
\author{Alex Avila, Bazalar Rommel,Konrad Trejo\\
\small{\texttt{\@aavilas@uni.pe, rbazalarc@uni.pe,btrejoc@uni.pe}}}
\maketitle



%Seccion Introduccion
\section{Introducción}

	Ante la necesidad de lograr un ranking de los equipos nacionales de fútbol, comprendido por los países que practican este deporte , el ranking FIFA no considera títulos ganados, pero si considera los partidos amistosos, lo cúal puede ser un problema pues para calcular ranking de países debe considerarse los títulos ganados y en cuánto a los amistosos algunos técnicos prueban los equipos, por lo tanto no debería entrar al ranking los partidos amistosos.

Con el fin de combatir este problema surge la idea de la creación de un nuevo
ranking  Basándonos en el método de PageRank, generado por Larry
Page y Sergey Brin, logramos generar un nuevo ranking que se destaca por la cantidad de títulos ganados, y no consideraremos los partidos amistosos.

Basándonos en un historial de los partidos mundialistas nos proponemos evaluar cómo
se comporta el nuevo método creado en comparación de otros rankings existentes y así
analizar si realmente logramos una mejora en el reflejo de la realidad.
Una vez obtenido el ranking, nos proponemos a, dado un partido, evaluar el
ranking de los países que lo disputan  así poder indicar
con que probabilidad saldrá victorioso el equipo con mejor posicionamiento.

Para lograr todo esto tendrémos que usar nuestros conocimientos en el curso de Probabilidades y también aprovechar nuestro conocimiento en el lenguaje R.

% La bibliografia se encuentra de el archivo 
 %bibliografia.bib adjunto en los\archivos

%Seccion estado del arte
\section{Estado del arte}
-Breve mención del aporte que otros artculos cientícos han realizado para este problema.

-Mención de al menos 2 artculos cientícos que mencionen el problema y las variantes
realizadas.

%seccion Diseño del experimento
\section{Diseño del experimento}
Descripción de los objetos, funciones y técnicas a utlizar.

\begin{itemize}
	\item
	Ejemplo item 1
	\item
	Ejemplo item 2
	\end{itemize}	
    %figura1.1
% Ejemplo de poner imagen   
%    \begin{figure}
%   \centering
%	\includegraphics[width=0.5\textwidth]{rotacion}
%	\caption{Rotaci\'on}\label{FI1_1}
%\end{figure}

%Subseccion Experimentos y resultados
\section{Experimentos y resultados}

\begin{itemize}
	\item
	Línea base: Reproducción de resultados reportados en un artculo cientíco
anterior.
	\item
	Evaluación del rendimiento de los modelos ensayados.
    \item
    Comparación de línea base y resultados propios.
	\end{itemize}	

\subsection{Ejemplo de subsecuencia}

Ejemplo de secuencias $X_{1}$,$X_{2}$, . . ., $X_{m}$

Ejemplo de sumatoria
\[O(
\sum_{k=1}^{n}p_{k}log(1/p_{k}))
\]
 Ejemplo de o grande
\begin{center}
 O(log($min_{i<j}$(|$x_{i}$| + $t_{i}$ + 2)))    
\end{center}

\section{Discusión}

-Interpretación de los resultados obtenidos.

-¿Cómo podría ser mejorado sus resultados?

%\begin{figure}

%	\includegraphics[width=0.5\textwidth]{zigzag.png}
%	\caption{Movimientos zig-zig y zig-zag}\label{FI1_2}
%\end{figure}


%\begin{figure}[h!]
%	\centering
%	\includegraphics[width=0.5\textwidth]{toques.png}
%	\caption{Gr\'afica de elementos del \'arbol versus recorridos por el algoritmo al buscarlos}\label{FI1_3}
%\end{figure}


\paragraph{Ejemplo de subtítulo}
Un conjunto de puntos que contenga a $ (x_i,i) $ es arb\'oricamente
satisfecho si y s\'olo si este corresponde a un \'arbol de b\'usqueda
binaria v\'alido para el input $ (x_1,...,x_m) $.

\section{Conclusiones y trabajos futuros}
Conclusiones del proyecto
\section{Bibliografía}

%Biliografia************************
%\bibliographystyle{apalike}
\nocite{*}
\bibliographystyle{apalike}
\bibliography{<1>,}

%Fin del documento
\end{document}